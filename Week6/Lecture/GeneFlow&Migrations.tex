\documentclass[11pt]{article}
\title{Gene Flow and Migrations}
\author{Jason Hodgson}
\date{8/11/2017}
\begin{document}

\maketitle
Gene flow homogenises different populations. \\
In gene trees, long branches can be because of inbreeding (e.g. dogs)\\
Use gene flow to work backwards to find ancestory and work out past migrations. Use SNP data, see madagascar example on slides.\\
D statistic: Used to test for evidence that the Neandertals are an outgroup tp modern humans. Look for overexpression of genes . Shared derived states should be similar if no gene flow from neandertals. If into one state then there should be an abundence of shared genes in that population. Results suggest some gene flow into Europeans, Asian (all populations tested except african populations). \\
\paragraph{Inferring how long ago migrations occured}
Can use linkage disequilibrium. (can use programming tool ALDER) Looking at popualtions around the world. Need to be careful as past papers show; paper saying migration back to Africa from middle east about 3000 years ago, looking at conditions at the time this probably wasn't possible. When test was repeated got same results but saw a problem that programme looked only at more recent admixture but when looked if there were older versions too and see that it's actually been seperating from other african populations 23000 years ago. This is much more likely. Overall, migration 23000 years ago, hunter gathers, small population accounts for most Eurasian ancestry. Compared to 3000 years ago migration, agriculturists, account for small part of Eurasian ancestry. But what about the impact of the Arabian slave trade? 7.2 million Africans transported to MENA, comparable in scale to Atlantic save trade, ~160million descendants of Atlantic slave trade in New World, clear descendants of Arabian slave trade rare in MENA so what happened to them? Probably a mixture of integration into the popualtiona and not having many offspring. 
\end{document}