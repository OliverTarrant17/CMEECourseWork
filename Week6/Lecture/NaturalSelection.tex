\documentclass[11pt]{article}
\title{Natural Selection}
\author{Jason Hodgson}
\date{8/11/2017}
\begin{document}
\maketitle
People need to adapt to there environment. Genetic drift is happening all the time so is used as null hypothesis. Natural selection doesn't happen all the time and affects one part of the genome. Natural selection increases the frequency of the gene being selected.
\paragraph{Linkage equilibrium and disequilibrium}
LD is the correlation between polymorphisms. LE is when there is no correlation between alleles. With time LD breaks down to LE.
\paragraph{Selective sweep}
When the advantage allele and it's alleles in LD become much more frequent. Complete when all haplotypes now have these alleles (no variation), incomplete  when just most of them do.
\paragraph{Natural selection causes}
Loss of genetic diversity around the selected site, increased LD around the selected site, an excess of the mst common allele, deficiency of intermediate frequency alleles, LD decays with time since selection began.
\paragraph{Lactase}
Lacatse expression high during infancy, expression stops around waning in all mammals except some humans. The persistence of the expression is strongly linked with location and ethnicity. Multiple mutations within the same region of the genome have occured at sperate times all of which express the lactase persistant phenotype. LD shows evidence of recent selection, selection began ~5000 years ago (approximately just after cattle were domesticated)
\paragraph{EDAR}
One of the most differentated parts of the genome is EDAR. Selection around here is very strong. Looked at in association with hair genes. East asians have almost a single haplotye for hair completely selceted. When doing tests in mice these results were also found. Not clear why one variation may  be advantageous over another. Possibly due to thermoregulation, sexual selection, maybe natural selection at different times. It remains unknown.
\paragraph{Evolution of resistance to Plasmodium vivax in Madagascar}
A less deadly form of Maleria, requires DARC protein on red blood cells to enter and infect cells, there is a Duffy-null mutation which results in the loss of DARC expression creating resistance. Distribution of P.vivax ad Duffy null. Duffy-null 100\% in sub Saharan africa, 0\% rest of the world. Is vivax malaria the selection pressure that drove fixation of the duffy-null allele in sub-saharan Africans? Admixed population in Madagascar, in highlands has 78\% Duffy-null (48\% expected) and 92\% Duffy-null on the south coast (67\% expected). Is this genetic drift or natural selection? \\
Use drift simualtions, amount of drift is a function of population size and number of generations, don't know effective population size, don't know maximum possible number of generation, simulations of genetic drift to define null hypothesis. \\
In all simulations can rule out genetic drift as strong enough force to change these alleles. Look across the genome, if genetic drift would expect to see a lot of diversity. None of simulations show enough diversity as shown in Duffy-null so suggests strong selction.\\
{\bf Conclusion} Duffy-null frequency higher than expected in Madagascar, frequency increase only explained by genetic drift under extreme demographic histories, observed genomic diversity greater than expected given extreme demographic histories, vivax malaria resistance 23\% to 61\%. Conclusion: natural selection for the Duffy-null allele.
\end{document}