\documentclass[11pt]{article}
\title{Population structure}
\author{Jason Hodgson}
\date{7/11/2017}
\begin{document}
\maketitle
\section{What is a population?}
A gorup of closely related organisms capable of mating with each other. Populations evolve. Population genetics concerned with allele frequencies in populations.\\
Interested in\\
\begin{itemize}
\item mutations 
\item SNP - single nucleotide 
polymorphism
\item Recombination\\
\end{itemize}
Hardy Weinburg equilibrium (usually null hypothesis for population genetics) assumes: Population infinitely large, no mutations, no gene flow, no selection, random mating \\
\\
Panmixia - Random mating (all individuals have an equal chance of mating with each potential mate), outbred (expected heterozygosity), expected relatedness between any two individuals equal to the population average, genetic
\\
\\
What happens when species are not panmictic? \\
Demes (more likely to mate with someone from your deme than from some other deme), more likely to share a most recent common ancestor with some than others. You have genetic ... structure\\
\\
Causes of genetic structure: non rndom mating, geographic features, distance, mate choice

\section{Why do we care about this?}
\begin{itemize}
\item population history
\item variation and geography
\item migration
\item connectivity
\item genotype/ penotype relationships
\item first step of speciation
\item conservation
\end{itemize}
\section{How does structure develope?}
\begin{itemize}
\item Allele frequency differences
\item evolution
\item genetic drift
\item natural selection
\item mutations
\item phenotypes
\end{itemize}

Most comes from genetic drift.


\end{document}
