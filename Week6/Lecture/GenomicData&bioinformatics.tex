\documentclass[11pt]{article}
\title{Genomic data and bioinformatics}
\author{Jason Hodgson}
\date{6/11/2017}
\begin{document}
\maketitle
\section{Data}
DNA data is easy to display visualise.\\
Biggest problem at the moment is that there is so mcuh data. \\
Up to 900 billion bases in a single experiment now.\\
Often presented in FASTA files: \\
\\
\\
>Taxa1\\
ATCGTAGCTACGTTTACCAGAAC\\
GGATCATTATTCTATATGCGGGA\\
etc.
\\
\\
FASTQ files: \\
DNA or RNA sequence data, includes a quality score.
\\
\\
VCF (variant call format)\\
Just variable sites mapped to a genome build. Meta data including some quality information.\\
\\
PLINK\\
SNP genotype data. Either 2 or 3 files per a dataset, a genotype file, a family file and a marker file.\\

\section{How to analyse the data}
Common software for genomic analysis in R: Genetic packages (popgen etc).\\
Advantages: easy to integrate with other statistical analysis. Excellent plotting capabilities.\\
Disadvantages: Extremely memry intensive. Often not possible with very large datasets. Slow.\\
\\
Because of this lots of stand alone programmes.
\\
PLINK: SNP data, designed for GWAS(genome wide association study) Powerful, fast, supports large datasets. Basic population genetics. Many models for testing genotype, phenotype associations.
\\
\\
Admixture: SNP or microsatellite data, model based method for inferring population structure and admixture proportions. Cam handle large datasets. 
\\
\\
Others: ALDER (admixture dating using LD), CHROMOPAINTER (local ancestry assignment), SAMSTOOLS (NGS data), USEARCCH (analysing metagenomic data), TREEMIX (inferring migrations) plus many more.
\\
\\
Work in UNIX - fast, simple and repeatable. \\
Script your analysis: automates it making it faster to run, makes it faster to repeat, makes a record of exactly what was done. REPEATABILITY.



\end{document}