\documentclass{pnastwo}
\usepackage{amssymb,amsfonts,amsmath}
%% For PNAS Only:
\contributor{Submitted to Proceedings
of the National Academy of Sciences of the United States of America}
\url{www.pnas.org/cgi/doi/10.1073/pnas.0709640104}
\copyrightyear{2014}
\issuedate{Issue Date}
\volume{Volume}
\issuenumber{Issue Number}
\begin{document}
\title{My Title}
\author{Some Name \affil{1}{Imperial College London, UK} \and
Some O. Name\affil{2}{University of Exeter, Penryn, Corwall, UK}}
\maketitle
\begin{article}
\begin{abstract}
Mind blowing abstract.
\end{abstract}
\keywords{term1 | term2 | term3}
%% Main text of the paper
\dropcap{I}n this work, we show how \LaTeX can be used to typeset a PNAS paper. Lorem ←-
ipsum dolor sit amet, consectetur adipiscing elit. Phasellus sodales consectetur ←-
lobortis. Proin tincidunt eros dapibus ipsum faucibus sed rhoncus augue mollis. In←-
lectus velit, interdum at adipiscing quis, imperdiet sed justo. Praesent commodo,←-
mi iaculis tincidunt mollis, sapien lectus aliquam neque, ac faucibus arcu est eu←-
sem. Ut non lacus lacus, eu suscipit odio. Aliquam erat volutpat. Vivamus dapibus←-
pretium nunc, et placerat turpis bibendum mollis. Fusce eu mi ut nulla accumsan ←-
viverra. In nulla tellus, ultrices ut venenatis nec, laoreet eget diam. ←-
Pellentesque aliquam facilisis ultricies. Vestibulum sollicitudin leo non neque ←-
vehicula a volutpat eros faucibus. Vestibulum nec lorem dui.
\begin{materials}
These are the materials and methods.
\end{materials}
\begin{acknowledgments}
-- text of acknowledgments here, including grant info --
\end{acknowledgments}
\end{article}
\end{document}
