\documentclass[11pt]{article}
\author{James Rosindell}
\title{High Performance Computing}
\date{28/11/2017}
\begin{document}
\maketitle
Look up how to log in to HPC\\
HPC is useful for embarrassingly parallel problems (graphics, simulations with multiple parameters) and Non embarrassingly parallel problems (fluid dynamics, a lot of the tasks run by a single problem).
\section{How do you parallelize your code?}
Imagine before using HPC you want to do one simulation then 3 parallel in HPC.\\ Important line of code:\\
$as.numeric(Sys.getenv("PBS_ARRAY_INDEX"))$
Using PC: for i in 1:10 do simulation(i)\\
$HPC: i <- as.numeric(Sys.getenv("PBS_ARRAY_INDEX"))do_simulation(i)&$\\
do_simulation <- function(i) { set random seed as i, select your simulation parameters, do your simulation, save your output in a file named ....i..., include a timer}.\\
Use different random number seeds when running parallel to ensure that they are independent.\\
Handling memory: Save your results in memory and then write to disk at the end. Output your code to a series of files. Write local code to read in your series of files automatically. Build a timer into your code. Test your code locally to know your memory and time requirements.
\section{Running code on a cluster}
Will be using: cx1 - a cluster of many ordinary computers. Access via a login node. Login node, your potal to cx1, Login.cx1.hpc.ic.ac.uk\\
Where is data stored? \\
\$Home- backed up main area 10GB\\
\$Work - for running jobs not backed up 150GB\\
\$TMPDIR - Ignore these temporary files but be careful you don't lose work in here\\
Need to write command in shell script to move files back from tmpdir to home. \\
Step1: Go to filezilla-project.org\\
Download FileZilla Client and install it\\
Open FileZilla and put in the following settings:\\ (see dropbox)
\section{For exercises}
Adapt code from yesterday to run the cluster for a much bigger ecological community size. Need to collect species abundance data as before and average over a large number of parallel simulations. Use burn in period and check the species abundance distribution periodically.
\end{document}