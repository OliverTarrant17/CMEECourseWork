\documentclass[11pt]{article}
\title{Introduction to HPC}
\date{27/11/2017}
\author{James Rosindell}
\begin{document}
\maketitle
\section{Ecological Neutral Theory}
\subsection{What is neutral theory?}
A theory in ecology is any ecological work that uses mathematical formulaes and computers. Ecological neutral theory assumes all individuals are ecological equivalent. It is about making some assumptions and seeing where that gets us.
\subsection{Misconceptions}
Term neutral model can be interchanged ith null model.\\
Neutral model assumes all species are the same (actually just the demographic properties of an individaual are independent of its species identity).\\
A model in which species are interchangable is neutral.\\
\section{Examples of neutral model}
\subsection{Voter model}
Pick a random neighbour, you take their view or they take yours. Keep going for a while and see what happens. \\
FInd impenetrable clumps following, edge effects, eventually everyone in each connected group holds the same view.
\section{What is neutral theory useful for?}
Understanding ecology and predicting the future.
\end{document}